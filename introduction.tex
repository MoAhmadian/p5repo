%*************************************************
\section{Introduction and motivation}
\label{sec:introduction}
%*************************************************

Two major security breaches \cite{weiss2015target, silver2014jpmorgan} illustrates the importance of security for  databases hosted by different organizations and by computer clouds. In November 2013, approximately 40 million records were stolen from an unencrypted database used by Target stores. The compromised information included personally identifiable information (PII) and credit cards data. Two months later, a cyber-attack directed at JP Morgan Chase, compromised PII records of 76 million households and 7 million small businesses, according to a SEC (Securities and Exchange Commission) report. The impact of these security breaches  on the privacy of millions individuals cannot be underestimated.

Nowadays, many enterprises use cloud Database As A Service (DBaaS) offered by major Cloud Service Providers (CSPs) to access the increasingly larger  number of cloud hosted databases.  The number of websites hosted by Amazon Web Services (AWS) increased from $6.8M$ in September 2012 to $11.6M$ in May 2013 showing $71\%$ upsurge \cite{Netcraft-AWS-Growth}. Furthermore, a $67\%$ annual growth rate is predicted for DBaaS by $2019$. 


Cloud users along outsource the storage and the management of their databases to the CSPs and expect their sensitive information to be confidential. Typical cloud database services guarantee availability and scalability, but the data confidentiality poses significant challenges in the face of new threats. Some of the threats emanate from insider attackers who have the ability to correlate information from multiple cloud databases. 

The term \emph{information leakage} describes circumstances when sensitive information is inadvertently made available to untrusted entities. In the context of this research, information leakage is the ability of an attacker to infer sensitive information either through multiple database searches or through any statistical analysis of cloud database queries. 

Information leakage occurs when a combination of low-risk information items results in inferring high-risk information and its risks are amplified when data is stored on computer clouds.  In on-premises data management systems, a \emph{Malicious Insider} (MI) has access to a few databases for a data inference attack, whereas in a cloud  an MI can access millions of datasets belonging to a large variety of enterprises and individuals. 

A motivating sequence of events show the effects of data correlation and implictly of information  leakage. In August 2006, AOL, a global on-line mass media corporation,  released search logs of over 650,000 users for research purposes. The data included searches conducted over a period of three months with user names changed to random ID numbers. The information found by analyzing the searches conducted by a user, made him/her uniquely identifiable. Correlating data released by AOL with publicly available datasets revealed additional private information about AOL users.

Information leakage can occur not only in unencrypted databases, but also in encrypted ones and can be exploited by both external and insider agents.  Data encryption provides data and query privacy but, contrary to the common belief, encrypted cloud data and encrypted queries are still vulnerable due to information leakage. A database can be encrypted by data owner before being outsourced to the cloud in such a way that client queries can still be processed on the transformed data. 

Encryption does not hide all information about the encrypted data. A malicious insider can infer sensitive information through cross-referencing with other databases in the data warehouse. Moreover, the collection name, the attribute name (or table, field name in RDBMS), the number of attributes involved in a query, and the query length often reveal sensitive information about the encrypted data. This type of attack on an encrypted database is categorized within the information leakage class.

A procedure to search an encrypted datasets involving a secure proxy, the \emph{ SecureNoSQL} \cite{Ahmadian2016SecureNoSQL} guarantees that a MI would never obtain the decryption keys. The proxy encrypts the queries from the clients and decrypts the query responses from the server. The process is completely transparent, without the clients being involved in encryption/decryption operation. The proxy construction assures that an MI could not access to the sensitive information. However, the MI can exploit leaked information from multiple databases to organize more extensive attacks and amplify leakage.

The discussion in this paper is restricted to  NoSQL databases with a flexible schema. A NoSQL database is a collection of documents $D={d_1,\dots,d_n}$. A document is a set of key-value pairs ${key_i, value_i}$, each representing an attribute of an object. Enforcing partial security mechanism which covers only a subset of attributes, may not provide comprehensive protection as protected information could be inferred using low-risk datasets hosted in the same cloud. 

The solution we propose is insertion of \emph{disinformation documents} into the collection. A  secure proxy like the one in \cite{Ahmadian2016SecureNoSQL} mediates the interaction between clients and the DBaaS server. The proxy intercepts the client queries, transfers them to the encrypted queries and passes them to the cloud DBaaS server which responds with a combination of valid and forged documents. 

Eventually, the proxy decrypts the query response and filters out the fake documents and forwards the desired document to the user's application.  A \emph{Selective Disinformation Document Padding} (SDDP) is proposed  to avoid the overhead of disinformation document padding in the dataset.   We also investigate \emph{encrypted data indexing} features for minimizing the query processing time of augmented ciphertext datasets.

The proposed solution consist of two parts: i) a secure proxy ii) cloud DBaaS for NoSQL database service. In the scheme there are three interested parties, the data owner, the cloud base NoSQL database service and database query issuer users which can be a business partners of the data owner. We assumes all participants are interacting in a public cloud eco-system. The proposed scheme is easily adaptable to the hybrid and community cloud environment where the security risk is lower than the public cloud.

\medskip

{\bf Problem statement and contributions.}  Information linkablity (a property of being linkable) is a distinctive aspect of data that makes  possible to link individual pieces of information from different sources. The cloud database service is a data warehouse, consists of thousands of hosted databases from various organizations. Therefore, information linkablity poses a new type of threat to the users�� privacy. The potential attacker could link low-risk pieces of information to extract the sensitive information about an entity. For example, a person connected metro card to the debit card for auto-refill purpose might lead to detect a correlation which could reveal his/her employer information through the direct deposit. Exposing some bank information could also disclose some further information such as health records. Potentially, a link between metro card and private information could be detected and learned.

In NoSQL database, all features of an entity are presented by attributes, this information linkablity occurs in form of attribute correlation amongst multiple databases. We suggest to selectively insert documents with false information in the dataset according to the statistical analytics in order to deliberately expand the volume of the information leakage to the untrusted database service provider to protect the real sensitive information. At the end, we filter out the fake documents and deliver the correct documents to the corresponding client.

The research reported in this paper covers information leakage for computer clouds. The main contributions of this paper are:

\begin{enumerate}[leftmargin=*]
\item 
A methods to quantify information leakage as a result of explicit and implicit attribute correlations in a cloud data warehouse.
\item
A selective disinformation document padding method to reduce information leakage with limited overhead.
\item 
A fast and scalable leakage assessment and parameter extraction algorithm for very large databases based on approximate query processing.
\end{enumerate}



