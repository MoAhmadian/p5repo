%*************************************************
\section{Introduction and Motivation}
\label{sec:introduction}
%*************************************************

Information leakage is the inadvertent disclosure of sensitive information. Information leakage enables an attacker to infer sensitive information either through multiple database searches or through a statistical analysis of database queries.  The threat posed by information leakage is amplified  as cloud data warehouses maintain a large number of hosted databases from various organizations.   

Information linkablity, the ability to link individual items of information from different sources poses a new type of threat to cloud users��.  An attacker could link low-risk pieces of information to extract the sensitive information. For example, a metro card connected to user's debit card for auto-refill could allow a cross-correlation revealing  sensitive financial information. Moreover, personal information from the financial record could be linked to the health record of the individual. 
 
Nowadays, many enterprises use DBaaS offered by major Cloud Service Providers (CSPs) to access the increasingly larger  number of cloud hosted databases  \cite{cloudcomputing17}.  A $67\%$ annual growth rate is predicted for DBaaS by $2019$.  CSPs guarantee availability and scalability, but the data confidentiality poses significant challenges in the face of new threats. Some of the threats emanate from insider attackers who have the ability to correlate information from multiple cloud databases. 

Information leakage  from unencrypted, as well as  encrypted cloud databases is possible.  A database can be encrypted before being outsourced to the cloud, yet allow  client queries to be processed on the transformed data. Data encryption provides data and query privacy but, contrary to the common belief, encrypted cloud data and encrypted queries are vulnerable to information leakage. The leakage can be exploited by external, as well as insider agents. 


Encryption does not hide all information about the encrypted data. A malicious insider can infer sensitive information through cross-referencing with other databases in the data warehouse. Moreover, the collection name, the attribute name (or table, field name in RDBMS), the number of attributes involved in a query, and the query length often reveal sensitive information about the encrypted data. 

A motivating sequence of events illustrate the effects of data correlation and, implicitly of information  leakage. In August 2006, AOL, a global on-line mass media corporation,  released search logs of over 650\,000 users for research purposes. The data included searches conducted over a period of three months with user names changed to random ID numbers. An analysis of the searches conducted by a user, made him/her uniquely identifiable. Correlating data released by AOL with publicly available datasets revealed additional private information about AOL users.

The discussion in this paper is restricted to  NoSQL databases with a flexible schema. A NoSQL database is a collection of documents $D={d_1,\dots,d_n}$. A document is a set of key-value pairs ${key_i, value_i}$, each representing an attribute of an object. Enforcing partial security mechanism which covers only a subset of attributes, may not provide comprehensive protection as protected information could be inferred using low-risk datasets hosted by the same cloud. 


A procedure to search an encrypted datasets involving a secure proxy, the \emph{ SecureNoSQL} \cite{Ahmadian2016SecureNoSQL} guarantees that an insider attacker would never obtain the decryption keys. The proxy encrypts the queries from the clients and decrypts the query responses from the server. The process is completely transparent, without the clients being involved in encryption/decryption operation. The proxy construction assures that an insider attacker could not access to the sensitive information. However, the insider attacker can exploit leaked information from multiple databases to organize more extensive attacks and amplify leakage.

The number of documents in  cloud databases  limits the  ability to analyze in real-time the dangers posed by information leakage. The alternative we  propose is based on random sampling and error estimation regarding information leakage. This  solution dramatically cuts the analysis time, especially, whenever the sample size is small enough to fit in the main memory of the database servers.  As expected, approximate measurements based on data sampling  exhibit different levels of errors. Sampling-based Approximate Query Processing (AQP) provides bounds on the error caused by sampling \cite{agarwal2014knowing, chatzikokolakis2009calculating}.


We propose insertion of \emph{disinformation documents} into the collection. A  secure proxy like the one in \cite{Ahmadian2016SecureNoSQL} mediates the interaction between clients and the DBaaS server. The proxy intercepts the client queries, transfers them to the encrypted queries and passes them to the cloud DBaaS server which responds with a combination of valid and forged documents. 

Eventually, the proxy decrypts the query response and filters out the fake documents and forwards the desired document to the user's application.  The Selective Disinformation Document Padding (SDDP)  is proposed  to avoid the overhead of disinformation document padding in the dataset.   We also investigate an Encrypted Data Indexing features for minimizing the query processing time of augmented ciphertext datasets. The contributions of this paper are:

\begin{enumerate}[leftmargin=*]
\item 
A method to quantify exact volume of information leakage as a result of explicit and implicit attribute cross-correlations in a cloud data warehouse.
\item
A selective disinformation document padding method to reduce information leakage with limited overhead.
\item 
A scalable leakage assessment and parameter extraction algorithm for very large databases based on approximate query processing.
\end{enumerate}