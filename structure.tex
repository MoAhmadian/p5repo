\usepackage{etex}
\usepackage{algorithm}% http://ctan.org/pkg/algorithms
\usepackage{algpseudocode}% http://ctan.org/pkg/algorithmicx
\usepackage{mathtools} % Bonus
\usepackage{hhline}
\usepackage{tikz}
\usetikzlibrary{trees}
\usetikzlibrary{shapes,arrows}
\usetikzlibrary{decorations.text}
\usepackage{tkz-graph}
% ========= UCF================================
\usepackage{graphicx}
\graphicspath{ {figures/} }
%==========Table ==============================
\usepackage{threeparttable}  
\usepackage{booktabs}
\usepackage{multirow}
\usepackage{bm}
\usepackage{float}
\usepackage{booktabs}
\usepackage{graphicx}
\usepackage[font=small,skip=10pt]{caption} % caption font and distance
\usepackage{subcaption}
\usepackage{paralist}
\usepackage{array} 
\usepackage{algpseudocode}
\usepackage{amsmath}
\usepackage{caption}
\usepackage{url}
\usepackage{multicol}
\usepackage{helvet}
\usepackage{courier}
\usepackage{amsmath}
\usepackage{amssymb}
\usepackage{xcolor}
\usepackage{rotating}
\usepackage{framed}
%==========Enume ===============================
\usepackage{enumitem}
\setlist[itemize]{leftmargin=*}
%==========Plot ================================
\usepackage{varwidth}% http://ctan.org/pkg/varwidth for formula indent

\usetikzlibrary{shapes,arrows,automata,shadows}
\usetikzlibrary{shapes.multipart}
\usetikzlibrary{positioning}%,calc}
\usepackage{pgfplots}
\pgfdeclarelayer{background}
\pgfdeclarelayer{foreground}
\pgfsetlayers{background,main,foreground}
\pgfplotsset{width=7cm,compat=1.8}
\usetikzlibrary{patterns}
\usetikzlibrary{matrix,fit}
%===========Listing ============================
\usepackage{listings}
\lstdefinelanguage{json}{
    basicstyle=\small\sffamily,
    showstringspaces=false,
    breaklines=true,
    frame=lines,
    backgroundcolor=\color{background},
}
%===========non italic math ====================
%\usepackage[LGRgreek]{mathastext} 
%===============================================
\newtheorem{definition}{Definition}
\definecolor{background}{HTML}{EEEEEE} %{EEEEEE}
%============Draw document symbol]==============
\makeatletter
\pgfdeclareshape{document}{
\inheritsavedanchors[from=rectangle] % this is nearly a rectangle
\inheritanchorborder[from=rectangle]
\inheritanchor[from=rectangle]{center}
\inheritanchor[from=rectangle]{north}
\inheritanchor[from=rectangle]{south}
\inheritanchor[from=rectangle]{west}
\inheritanchor[from=rectangle]{east}
% ... and possibly more
\backgroundpath{% this is new
% store lower right \usepackage{float}
in xa/ya and upper right in xb/yb
\southwest \pgf@xa=\pgf@x \pgf@ya=\pgf@y
\northeast \pgf@xb=\pgf@x \pgf@yb=\pgf@y
% compute corner of ‘‘flipped page’’
\pgf@xc=\pgf@xb \advance\pgf@xc by-20pt % this should be a parameter
\pgf@yc=\pgf@yb \advance\pgf@yc by-20pt
% construct main path
\pgfpathmoveto{\pgfpoint{\pgf@xa}{\pgf@ya}}
\pgfpathlineto{\pgfpoint{\pgf@xa}{\pgf@yb}}
\pgfpathlineto{\pgfpoint{\pgf@xc}{\pgf@yb}}
\pgfpathlineto{\pgfpoint{\pgf@xb}{\pgf@yc}}
\pgfpathlineto{\pgfpoint{\pgf@xb}{\pgf@ya}}
\pgfpathclose
% add little corner
\pgfpathmoveto{\pgfpoint{\pgf@xc}{\pgf@yb}}
\pgfpathlineto{\pgfpoint{\pgf@xc}{\pgf@yc}}
\pgfpathlineto{\pgfpoint{\pgf@xb}{\pgf@yc}}
\pgfpathlineto{\pgfpoint{\pgf@xc}{\pgf@yc}}
}
}

%============Margin ==================================
\setlength{\oddsidemargin}{0.005in}
\setlength{\textwidth}{6.5in}
\setlength{\topmargin}{-0.8in}
\setlength{\textheight}{8.9in}
%=====================================================
\tikzset{
box1/.style={draw=black, thick, rectangle,rounded corners, minimum height=3cm, minimum width=3cm},
box2/.style={draw=black, thick, rectangle, minimum height=4.5cm, minimum width=4.5cm},
}
%============ Arrow decoration =======================
\usetikzlibrary{arrows, decorations.markings}
% for double arrows a la chef
% adapt line thickness and line width, if needed
%\tikzstyle{vecArrow} = [thick, decoration={markings,mark=at position
%   1 with {\arrow[semithick]{open triangle 60}}},
%   double distance=1.4pt, shorten >= 5.5pt,
%   preaction = {decorate},
%   postaction = {draw,line width=1.4pt, white,shorten >= 4.5pt}]
%\tikzstyle{innerWhite} = [semithick, white,line width=1.4pt, shorten >= 4.5pt]
%========================================================
%appendix
\usepackage[title,titletoc,toc]{appendix}

\renewcommand{\algorithmicensure}{\textbf{Output:}}



\usepackage{color}   %May be necessary if you want to color links
\usepackage{hyperref}
\hypersetup{
    colorlinks=true, %set true if you want colored links
    linktoc=all,     %set to all if you want both sections and subsections linked
    linkcolor=black,  %choose some color if you want links to stand out
	citecolor=black,
    urlcolor=blue,
}
%========================================
