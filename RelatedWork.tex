%****************************************************************************
\section{Related Work}
\label{sec:relatedWork}
%*****************************************************************************

Query processing for outsourced databases is reported in \cite{popa2011cryptdb,Ahmadian2016SecureNoSQL,7889242}. Sometimes, data is encrypted before a database is stored on a cloud; database queries to the cloud database are also encrypted. For instance, CryptDB \cite{popa2011cryptdb} and SecureNoSQL \cite{Ahmadian2016SecureNoSQL} can be used for encrypted SQL and NoSQL cloud databases, respectively. 

These two systems do not require modifications of database services, encrypted data is processed identically as the plaintext data. Database optimizations such as multi-layer indexing and cache and file management are applied to encrypted databases without any modifications. However, no existing systems address information leakage of encrypted data.

Full Homomorphic Encryption (FHE) allows computations to be performed on cyphertext. Decryption of the encrypted results of these operations are identical with the results of operations performed on the corresponding plaintext. FHE is theoretically feasible \cite{gentry2010computing}, but impractical due to the enormous overhead involved.

Order Preserving Encryption (OPE) is a deterministic cryptosystem where encryption of numeric data preserves the ordering of the plaintext data. Aggregate queries such as comparison, min, and max can be executed on OPE-encrypted datasets. OPE offers less protection than FHE and leaks critical information about the plaintext data \cite{ahmadian2014security, boldyreva2009order}. 

Searchable encryption methods such as Oblivious RAM (ORAM) \cite{liu2014search, ostrovsky1990efficient} provide an acceptable level of security. However, the efficiency and the high computational cost, as well as the excessive communication costs between the clients and the server make this method impractical \cite{goldreich1996software}.

Inference attacks against CryptDB are discussed in \cite{naveed2015inference}. Deterministic and OPE cryptosystems leak critical information such as frequency and order of the original data and enable attackers to extract sensitive information. Two cloud information leakage prevention methods have been proposed: (i) restrict the sequence of queries or the query processing time; and (ii) insert fake documents.

A quantitative characterization of correlation-based information leakage, formulated through the capacity of a $(n,q)-leakage$ channel is studied using restriction on the number of query or query processing time \cite{4221659}. The n-channel capacity is defined as the maximum possible depth of query-able information for any user. 

The capacity of a n-leakage channel, is defined as the probability of accessing a specific sensitive document in $n$ trials. In this approach, a group of documents form a chain with track of key-value pairs, and attacker can locate the head of the chain. By following the footprints, the rest of sensitive information can be accessed. The attacker is not only constrained by the number of trials, but also by the time necessary to achieve his objectives.

Two approaches to manage information leakage are used. The first builds a query dependency graph and a proxy prevents query execution when the accumulated leaked information reaches the maximum level of disclosure. This method is vulnerable to collaborative attacks and it is not effective against insider an attacker \cite{4221659}.

The second approach propose insertion of fake documents in the database. The fake information is expected to mislead the attacker by providing multiple values to an attribute. This solution drastically increases the number of total documents, the search time, and the communication costs\cite{whang2010managing}. However, multi-level indexing could reduce the query processing time in this case.

\medskip

\noindent {\bf Cloud Data Encryption Schemes.} Cloud data can be in three states store, transit, or process and any comprehensive data security mechanism must protect data in any of these three states. The communication channels can be secured using standard HTTP over the SSL (Secure Socket Layer) communication protocol. 

Most CSPs provide an API for the web service that enables developers to use both the standard HTTP and the secure version of the HTTPS protocol. The security requirements of data in transit state can be fully satisfied using HTTPS for communication with cloud. The endpoint authentication feature SSL makes it possible to ensure that the clients are communicating with an authentic cloud server.

The basic idea is to encrypt the data before uploading it to CSP. However, the data should be decrypted by the cloud server before getting processed. In other words, the data owner should disclose decryption key to the server in order to decrypt the data before performing any required operation. The problem is when the decryption key is compromised, the data confidentiality would be affected. Therefore, in the cloud computing environment new cryptosystems are required. 

\smallskip


\noindent \textbf{\textit{Random (RND):}} In a RND type encryption scheme, a message is coupled with a key $k$ and a random Initial Vector (IV). This scheme is non-deterministic, encryption of the same message with the same key yields different ciphertext. This randomness provides the highest level of security. 

The randomness property is achievable with different encryption algorithms. Advanced Encryption Standard (AES) with Cipher Block Chaining (CBC) mode is used for RND encryption. AES is a symmetric block cipher algorithm with a key size of 128,192 or 256 bits and with a block size of 128 bits. RND type schemes are semantically secure against chosen plaintext attacks and hides all kinds of information about ciphertext. As a result, RND scheme does not allow any efficient computation on the ciphertext. Equation \ref{eq1} describes the encryption and decryption of a block cipher in CBC mode.

\begin{equation} 
\label{eq1}
\begin{aligned}
for \quad j=2 \ldots n ;\qquad \begin{cases}
C_1 = E_k(P_1 \oplus IV), \qquad P_1 &= IV \oplus D_k(C_1) \\
C_j = E_k(P_j \oplus C_{j-1}), \quad 
P_j &= C_{j-1} \oplus 
D_k(C_j)\\ 
\end{cases}\\
\end{aligned}
\end{equation}

\medskip

\noindent \textbf{\textit{Deterministic (DET):}} A DET encryption scheme produces the same ciphertext for an identical pair of plaintext and key. Block ciphers in Electronic Code Book (ECB) mode, with a constant initialization vector are deterministic (DET). Deterministic encryption scheme preserves equality, therefore, the frequencies information of the searched keywords leaks to the third party. 

AES scheme in ECB mode is used for DET encryption over document-oriented NoSQL databases. DET scheme enables server to process pipeline aggregation stages such as group, count, retrieving distinct values and equality match \footnote{Equality matches over specific common fields in an embedded document will select documents in the collection where the embedded document contains the specified fields with the specified values.} on the fields within an embedded document. The embedded document can maintain the link with the primary document through application of DET encryption. Equation \ref{eq2} describes the encryption and decryption operation in a DET.

\begin{equation} \label{eq2}
\begin{aligned}
for \quad j=1, \ldots, n ; \qquad \qquad C_j & = E_k(P_j); \qquad \quad 
& P_j & = D_k(C_j) 
\end{aligned}
\end{equation}
\normalsize Where: $E_k$ is the encryption algorithm, $D_k$ is the Decryption algorithm, $k$ is the secret key, $P$ is a block of plaintext data and $C$ is a block of ciphered data.


\normalsize

\medskip

\noindent \textbf{\textit{Order-Preserving Encryption (OPE):}} OPE projects the order relation between plaintext data elements and their ciphertext values. OPE leaks the order of ciphertext, so it supports a lower degree of security. Even in Modular Order-Preserving Encryption (MOPE) \cite{Mavroforakis:2015:MOE:2723372.2749455} which is an extension to the basic OPE for security improvement, there is information leakage. An efficient inequality comparisons on the encrypted data elements can be performed by applying OPE which supports range queries, comparison, min and max on the ciphertext. We use the algorithm introduced in \cite{boldyreva2009order} and implemented in \cite{ahmadian2014security} for cloud environment. Equation \ref{eq3} shows the preservation of order relation of plaintext in the ciphertext.

\begin{equation} \label{eq3}
\begin{aligned}
\forall x, ~y ~|~ x,~y~ \in {Data~ Domain} \qquad x < y \implies OPE_k(x) < OPE_k(y)
\end{aligned}
\end{equation}

\normalsize Where $OPE_k$ is key-based OPE.


\medskip 

\noindent \textbf{\textit{Additive Homomorphic Encryption (AHOM):}} AHOM allows the server to conduct computations on ciphertext with the final result that is decrypted at the proxy. In spite of sustained research efforts \cite{gentry2009fully, brakerski2014efficient} of the Fully Homomorphic Encryption (FHE), there is no efficient FHE, except for some limited operations. AHOM is formulated in Equation \ref{eq4}. It should be noted that $m_1, m_2$ are messages to be encrypted where $m_1 , m_2 \in \mathbb Z_{n}$. $r_1, r_2$ are randomly selected and $r_1, r_2 \in \mathbb Z^{*}_{n}$. In other words, the product of two ciphertext messages decrypt to the sum of their corresponding plaintext.

\begin{equation} \label{eq4}
\begin{aligned}
D_k(E_k(m_1, r_1) \times E_k(m_2, r_2) ~mod ~n^2) & = m_1 + m_2~ (mod~ n) 
\end{aligned}
\end{equation}

In this paper whenever the summation and multiplication operations are carried out over encrypted data, we applied Paillier \cite{paillier1999public} scheme which is an instance of AHOM.

Data granularity indicates the level of detail, the more detail, the higher granularity level.
Encryption can be applied for datasets at various levels of granularity from high granularity data, like atomic level data element to low level granularity in aggregated atomic data items. For data store encryption, it is conceivable to have different encryption granularity levels according to the corresponding data granularity. 

The higher level of encryption granularity, the higher the information leakage. For example, encryption of a single attribute leaks frequency information, while encryption of the entire document and collection as a single unit leaks less information. In the work reported in this paper we use different encryption granularity matching the data granularity level.