%*************************************************
\section{Metric for explicit information leakage}
\label{app:primaryMetric}
%*************************************************

We define a notion to show what portion of disinformation is valid, by adapting {\it precision}, {\it recall} and {\it F-Measure} from information retrieval and binary classification literature \cite{davis2006relationship}. In general, precision\footnote{Indicates how many relevant items are selected }, recall\footnote{Indicates how many selected items are relevant} and F-Measure\footnote{ Harmonic mean of precision and recall} are simple metrics for quantifying valid information found inside any given document. The precision takes all attributes of document $d$ into account fo find the number of those attributes which are member of the super-document of attributes. The higher precision value means more common attributes that increases the probability of being merged into the super-document provided that at least of the attributes is from identifier class or group of them are semi-identifier. In this case, the uncommon attributes are considered as leaked information. The recall measures the strength of merge capability of a pair of documents (see Equation \ref{eq:leakageMetrics}). Precision of document $d$ denoted as $P_d$ which is a fraction of information in document $d$ with respect to super-document  $\delta$ containing attributes that their authenticity is established. \textit{Recall} and \textit{F-Measure} for document $d$ are denoted as $R_d$ and $F_{1}$ respectively:
\begin{equation} 
\label{eq:leakageMetrics}
\begin{aligned}
P_d=\frac{\left\lvert d \cap \delta \right\rvert}{\left\lvert d \right\rvert},\qquad 
R_d=\frac{\left\lvert d \cap \delta \right\rvert}{\left\lvert \delta \right\rvert},\qquad
F_{1}=\frac{2 \times P_d \times R_d}{P_d + R_d}
\end{aligned}
\end{equation}

The concepts of precision and recall based on document and super-document are demonstrated in more detail with an example. By definition, the super-documents includes all the accurate attributes ($A_i$) related to an object of interest. The document, on the other hand, includes inaccurate and/or fake attributes ($\mathcal{F}_j$). The fake documents are actually the noisy documents (disinformation) which is added to the collection in order to minimize the information leakage. A small fraction of document includes accurate attributes and the remaining are false. If the common attributes between the document and super-document are minimized, the lower information leakage is warranted. Consider Figure \ref{fig:PrecisionRecall} in which super-document $\delta$, includes $8$ accurate attributes and document $\delta$ includes $5$ (i.e. $2$ accurate and $3$ fake) attributes.\\

%Figure in the tex file
\begin{figure}[h]
\centering
\begin{tikzpicture}
\node[box2, fill=white, fill opacity=0.2] (c2) at (0,0) {};
\node[box2, fill=gray!20, fill opacity=0.3] (c1) at (4.5,0) {};
\fill[blue!20,fill opacity=0.2] (2.25,0) circle (2cm);
\draw (2.25,0) circle (2cm) node [above] {};
\node[label={\large Relevant Attributes}]  at (-0.15,2.2) {};
\node[label={\large Irrelevant Attributes}]  at (4.5,2.2) {};

\fill[black] (-1.5,1.0) circle (0.25cm);
\node[label={\small $A_6$}]  at (-1.5,1.0) {};

\fill[black] (0,1.0) circle (0.25cm);
\node[label={\small $A_2$}]  at (0,1.0) {};

\fill[black] (1,0.5) circle (0.25cm);
\node[label={\small $A_3$}]  at (1,0.5) {};

\fill[black!50] (3,1.0) circle (0.25cm);
\node[label={\small $\mathcal{F}_1$}]  at (3,1.0) {};

\fill[black!50] (5.5,1.0) circle (0.25cm);
\node[label={\small $\mathcal{F}_4$}]  at (5.5,1.0) {};

\fill[black] (-1.5,0.0) circle (0.25cm);
\node[label={\small $A_4$}]  at (-1.5,0.0) {};

\fill[black] (-0.4,0) circle (0.25cm);
\node[label={\small $A_5$}]  at (-0.4,0.0) {};

\fill[black!50] (3.7,0.0) circle (0.25cm);
\node[label={\small $\mathcal{F}_2$}]  at (3.7,0.0) {};

\fill[black!50] (5,0.0) circle (0.25cm);
\node[label={\small $\mathcal{F}_5$}]  at (5,0.0) {};

\fill[black] (-1.5,-1) circle (0.25cm);
\node[label={\small $A_8$}]  at (-1.5,-1) {};

\fill[black] (0,-1) circle (0.25cm);
\node[label={\small $A_7$}]  at (0,-1) {};

\fill[black] (1,-1) circle (0.25cm);
\node[label={\small $A_1$}]  at (1,-1) {};

\fill[black!50] (3.0,-1) circle (0.25cm);
\node[label={\small $\mathcal{F}_3$}]  at (3.0,-1) {};

\fill[black!50] (5.5,-1) circle (0.25cm);
\node[label={\small $\mathcal{F}_6$}]  at (5.5,-1) {};

\end{tikzpicture}
\caption{An example to demonstrate Precision and Recall}
\label{fig:PrecisionRecall}
\end{figure}

\begin{align*}
\delta&=\{A_1,~A_2,~A_3,...,~A_8\}\\
\rho&=\{A_1,~A_3,~\mathcal{F}_1,~\mathcal{F}_2,~\mathcal{F}_3\}\\
\rho \cap \delta &=A_1,~A_3; \qquad \left\lvert \rho \cap \delta \right\rvert=2\\
P_{\rho}&=\frac{\left\lvert \rho \cap \delta \right\rvert}{\left\lvert \rho \right\rvert} = \frac{2}{5},\qquad 
R_{\rho}=\frac{\left\lvert \rho \cap \delta \right\rvert}{\left\lvert \delta \right\rvert}= \frac{2}{8},\qquad
F_{1}=\frac{2 \times P_{\rho} \times R_{\rho}}{P_\rho + R_\rho}= \frac{0.2}{0.65}=0.307
\end{align*}
